\documentclass[12pt]{article}
\usepackage{url,setspace,amsmath}
\usepackage{graphicx}
\setlength{\oddsidemargin}{-8mm}
\setlength{\evensidemargin}{0mm}
\setlength{\textwidth}{175mm}
\setlength{\topmargin}{-5mm}
\setlength{\textheight}{240mm}
\setlength{\headheight}{0cm}
\setstretch{1}

\begin{document}
\title{\vspace{-2cm}Documentation for particledynamics1D: code for particle dynamics in a 1D domain}
\author{S.~S.~Mogre, E.~F.~Koslover}
\date{Last updated \today}
\maketitle

The code in this package will run a simulation of particle dynamics in a one-dimensional domain.

%\tableofcontents
%\newpage

\section{Compilation Instructions}
To compile and run the program, you will need the following:
\begin{itemize}
\item A compiler capable of handling Fortran90.
The code has been tested with the gfortran compiler. The default compiler is gfortran.
\item BLAS and LAPACK libraries installed in a place where the compiler knows to look for them
\item Optionally: Matlab to visualize output data
\end{itemize}

The code has been tested on Ubuntu Linux. 

To compile with gfortran, go into the \path=source= directory. Type \verb=make=.
To compile with any other compiler that can handle Fortran90, type
\begin{verbatim}
make FC=compiler
\end{verbatim}
substituting in the command you usually use to call the compiler. 

If the compilation works properly, the executable \path=partdynamics1D.exe= will appear in the main directory.

\section{Usage Instructions}
To run the program in the main directory, type:
\begin{verbatim}
./partdynamics1D.exe suffix
\end{verbatim}

Here, \verb=suffix= can be any string up to 100 characters in length. 
The program reads in all input information from a file named
\path=param.suffix= where, again, \verb=suffix= is the command-line
argument. If no argument is supplied, it will look for a file named
\path=param=. If the desired parameter file does not exist, the
program will exit with an error. You can supply multiple suffixes to read in multiple parameter files.

The parameters in the input file are given in the format `{\em KEYWORD} value' where the possible keywords and values are described
in Section \ref{sec:keywords}. Each keyword goes on a separate
line. Any line that starts with "\#" is treated as a comment and
ignored. Any blank line is also ignored. The keywords in the parameter
file are not case sensitive. For the most part, the order in which the
keywords are given does not matter. All parameters have default
values, so you need only specify keywords and values when you want to
change something from the default.


\section{Example for a Quick Start}
\subsection{Example}
An example parameter file (\verb=param.example1=) is provided. This will run a simulation for the dynamics of a system with 2 different types of protein-carrying particles. The particle types start at opposite ends of a 1D domain, moving towards the other end at a constant velocity. Particle type 1 undergoes stochastic reversals of direction. Particles of different types can fuse with each other. Run the example with 
\begin{verbatim}
./partdynamics1D.exe example
\end{verbatim}

Use the script \verb=checkexample.m= to visualize the movement and protein flux into one end of the domain. (NOT IMPLEMENTED)
%It should look something like this:

%\centerline{\includegraphics[width=0.7\textwidth]{example_MSD.eps}}

%\subsection{Example 2}
%Example 2 runs a simulation with one particle type performing stochastic direction reversals. Use the script \verb=example2.m= to visualize particle movement and protein flux.

% ---------------------------------------------------------

\section{Keyword Index}
\label{sec:keywords}
The code will attempt to read parameters out of a file named \path=param.suffix= where ``suffix'' is the command line argument. If no command line arguments are supplied, it will look for a file named \path=param=. If multiple arguments are supplied, it will read multiple parameter files in sequence.

The parameter file should have one keyword per line and must end with a blank line. All blank lines and all lines beginning with \# are ignored. For the most part, the order of the lines and the capitalization of the keywords does not matter. All keywords except {\em ACTION} are optional. The default values for each parameter are listed below. If a keyword is supplied, then values may or may not be needed as well. Again, the required and optional value types are listed below. 

Keywords and multiple values are separated by spaces. 

When reading the parameter file, lines longer than 500 characters will be truncated. To continue onto the next line, add ``+++'' at the end of the line to be continued.
No individual keyword or  value should be longer than 100 characters.

Floating point numbers can be formated as $1.0$, $1.1D0$, $10e-1$, $-1.0E+01$, etc., where the exponential notation specifier must be D or E (case insensitive). Integer numbers can also be specified in exponential notation without decimal points (eg: 1000 or 1E3). Logical values can be specified as T, F, TRUE, FALSE, 1, or 0 (with 1 corresponding to true and 0 to false).

The length units are set with respect to the domain length (\textit{DOMLEN}). Time units are set using a combination of velocity (\textit{VEL}) and domain length.

\begin{itemize}
%
\item {\it ACTION}
  \begin{itemize}
    \item  value: 1 string of at most 20 characters; no default
    \item This keyword sets the overall calculation performed by the program (see Sec.\ref{sec:tasks})
    \item Possible values are: RUNDYNAMICS
    \item Input format: ACTION string
    \item `string' can contain up to 20 characters.
  \end{itemize}
%
\item {\it ACIDIFICATIONRATE}
\begin{itemize}
	\item  type: float; size: \textit{NTYPE}; default: 0
	\item Rate of acidification/enzyme acquisition (used for vATPase pump model)
	\item Input format: ACIDIFICATIONRATE\quad$k_{1}\quad k_{2}\quad...\quad k_{NTYPE}$
	\item $k_i$ is the acidification rate for particle type $i$
\end{itemize}
%
\item {\it BUFLEN}
  \begin{itemize}
    \item  type: float; size: 1; default: 0
    \item Buffer length to determine an encounter event between particles
    \item Input format: BUFLEN value
  \end{itemize}
%
\item {\it DIFFCONST} (not implmented)
  \begin{itemize}
    \item  type: float; size: \textit{NTYPE}; default: 0
    \item Diffusion coefficient for each particle type. If only one value is entered, all particles types will have the same diffusion coefficient.
    \item Input format: DIFFCONST\quad$D_{1}\quad D_{2}\quad...\quad D_{NTYPE}$
    \item $D_{i}$ is the diffusion coefficient for particle type $i$.
  \end{itemize}
%
\item {\it DOMLEN}
	\begin{itemize}
	 	\item  type: float; size: 1; default: 1
	 	\item Length of the simulation domain.
	 	\item Input format: DOMLEN value
	\end{itemize}
%
\item {\it DELT}
  \begin{itemize}
    \item  type: float; size: 1; default: 1D-1
    \item Time-step for dynamics
    \item Input format: DELT value
  \end{itemize}
%
\item {\it DOFUSION}
  \begin{itemize}
   	\item  type: logical; size: 1; default: F
   	\item Determines whether particles undergo fusion.
   	\item Input format: DOFUSION value
  \end{itemize}
%
\item {\it KFUSE}
	\begin{itemize}
		\item  type: float; size: \textit{NTYPE} $\times$ \textit{NTYPE}; default: 0
		\item Rate of fusion between particle types.
		\item Input format: KFUSE\quad$i$\quad$k_{i,1}\quad k_{i,2}\quad...\quad k_{i,NTYPE}$
		\item $k_{i,j}$ is the fusion rate between particle types $i$ and $j$
		\item Multiple lines should be used for different values of $i$.
	\end{itemize}
%
\item {\it KPROD}
	\begin{itemize}
		\item  type: float; size: \textit{NTYPE}; default: 0
		\item Rate of production for all particle types. If only one value is entered, all particles are produced at that rate. 
		\item Input format: KPROD\quad$k_{1}\quad k_{2}\quad...\quad k_{NTYPE}$
		\item $k_{i}$ is the rate at which particle type $i$ is produced.
	\end{itemize} 
%
\item {\it KREV}
	\begin{itemize}
		\item  type: float; size: \textit{NTYPE}; default: 0
		\item Rate of direction reversal for all particle types. 
		\item Input format: KREV\quad$k_{1}\quad k_{2}\quad...\quad k_{NTYPE}$
		\item $k_{i}$ is the rate at which particle type $i$ reverses movement direction.
	\end{itemize}
%
\item {\it KSTOP}
	\begin{itemize}
		\item  type: float; size: \textit{NTYPE}; default: 0
		\item Rate of pausing for all particle types. If only one value is entered, all particle types are assigned that pause rate. 
		\item Input format: KSTOP\quad$k_{1}\quad k_{2}\quad...\quad k_{NTYPE}$
		\item $k_{i}$ is the rate at which particle type $i$ pauses moving.
	\end{itemize}
%
\item {\it MAXNFUSE}
	\begin{itemize}
		\item  type: integer; size: \textit{NTYPE}; default: 1
		\item Number of fusion events needed to saturate particle types. If only one value is entered, all particle types are assigned that value.
		\item Input format: MAXNFUSE\quad$N_{1}\quad N_{2}\quad...\quad N_{NTYPE}$
		\item $N_{i}$ is the number of fusion events needed to saturate a particle of type $i$.
	\end{itemize}  
%
\item {\it NPART}
	\begin{itemize}
		\item  type: integer; size: \textit{NTYPE}; default: 1
		\item Initial number of particles of each type. If only one value is entered, all particle types are assigned that value.
		\item Input format: NPART\quad$N_{1}\quad N_{2}\quad...\quad N_{NTYPE}$
		\item $N_{i}$ is the initial number of particles of type $i$.
	\end{itemize}   
%
\item {\it NPROT}
	\begin{itemize}
		\item  type: integer; size: \textit{NTYPE}; default: 1
		\item Number of protein types within each particle type. If only one value is entered, all particle types are assigned that value.
		\item Input format: NPROT\quad$N_{1}\quad N_{2}\quad...\quad N_{NTYPE}$
		\item $N_{i}$ is the number of protein types in particle type $i$.
	\end{itemize}
%
\item {\it NREG}
\begin{itemize}
	\item  type: integer; size: 1; default: 0
	\item Number of special regions within the domain.
	\item Input format: NREG\quad value
\end{itemize} 
%
\item {\it NSTEPS}
\begin{itemize}
	\item  type: integer; size: 1; default: 0
	\item Number of simulation steps.
	\item Input format: NSTEPS\quad value
\end{itemize} 
%
\item {\it NTRIALS}
\begin{itemize}
	\item  type: integer; size: 1; default: 0
	\item Number of independent simulation trials.
	\item Input format: NTRIALS\quad value
\end{itemize} 
%
\item {\it NTYPE}
\begin{itemize}
	\item  type: integer; size: 1; default: 0
	\item Number of particle types. Maximum 5 types allowed.
	\item Input format: NTYPE\quad value
\end{itemize}
%
\item {\it OUTFILE}
	\begin{itemize}
		\item 1 string; default: *.out
		\item File to which output is written. Can also be specified in OUTPUT.
	\end{itemize} 
%
\item {\it OUTPUT}
\begin{itemize}
	\item 1 optional integer, 1 optional string; defaults: 1, *.out
	\item Information about writing output to file.
	\item integer: how often to write output; string: output file (* is replaced with suffix)
\end{itemize}
%
\item {\it PFUSE}
\begin{itemize}
	\item  type: float; size: \textit{NTYPE} $\times$ \textit{NTYPE}; default: 0
	\item Probability of fusion between particle types. If this option is used, probabilities will be used instead of rates to determine fusion events.
	\item Input format: PFUSE\quad$i$\quad$p_{i,1}\quad p_{i,2}\quad...\quad p_{i,NTYPE}$
	\item $p_{i,j}$ is the fusion rate between particle types $i$ and $j$
	\item Multiple lines should be used for different values of $i$.
\end{itemize}
%
\item {\it PRAD}
\begin{itemize}
	\item  type: float; size: \textit{NTYPE}; default: 1D-2
	\item Rate of direction reversal for all particle types. If only one value is entered, all particles are assigned that reversal rate. 
	\item Input format: PRAD\quad$r_{1}\quad r_{2}\quad...\quad r_{NTYPE}$
	\item $r_{i}$ is the radius of particles of type $i$.
\end{itemize}
%
\item {\it PRINTEVERY}
\begin{itemize}
	\item  type: integer; size: 1; default: 0
	\item How often simulation status is displayed on the console.
	\item Input format: PRINTEVERY\quad value
\end{itemize}
%
\item {\it PROTCONV}
\begin{itemize}
	\item  type: float; size: \textit{NPROT} $\times$ \textit{NPROT}$\times$\textit{NTYPE}; default: 0
	\item Rates of conversion between different protein types within different particle types.
	\item Input format: PROTCONV\quad$i$\quad$j$\quad$k_{i,j,1}\quad k_{i,j,2}\quad...\quad k_{i,j,NTYPE}$
	\item $k_{i,j,k}$ is the conversion rate for protein $j$ to protein $i$ in particle type $k$.
	\item Multiple lines should be used for different values of $i$ and $j$.
\end{itemize}
%
\item {\it REGPOS}
\begin{itemize}
	\item  type: float; size: \textit{NREG}; default: -1
	\item Positions of special regions within the domain. A negative value results in equally spaced regions.
	\item Input format: REGPOS\quad$x_{1}\quad x_{2}\quad...\quad x_{NREG}$
	\item $x_{i}$ is the position of region $i$.
\end{itemize}
%    
\item {\it RNGSEED}
\begin{itemize}
	\item 1 integer; default: 0
	\item seed for random number generator
	\item value of 0 will seed with system time in milliseconds
	\item value of -1 will use the last 5 characters in the suffix
	\item value of -2 will use the last 4 charactes in the suffix and the millisecond time
	\item other positive value: the seed is used directly for repeatable simulations (should be positive)
\end{itemize}
%
\item {\it SNAPFILE}
\begin{itemize}
	\item 1 string; default: *.snap.out
	\item File to which snapshot of particles is written. Can also be specified within SNAPSHOTS.
\end{itemize}
%
\item {\it SNAPSHOTS}
  \begin{itemize}
    \item 1 optional integer, 1 optional string; defaults: 1, *.snap.out 
    \item Dump snapshots over the course of the simulation.
    \item integer: how often to dump snapshots; string: snapshot file (* is replaced with suffix).
   \end{itemize}
%
\item {\it STARTPOS}
\begin{itemize}
	\item  type: float; size: \textit{NTYPE}; default: 0
	\item Position at which particles of different types are produced.
	\item Input format: STARTPOS\quad$x_{1}\quad x_{2}\quad...\quad x_{NTYPE}$
	\item $x_{i}$ is the starting position for particles of type $i$.
\end{itemize}
%
\item {\it STARTPROT}
\begin{itemize}
	\item  type: float; size: \textit{NTYPE} $\times$ \textit{NPROT}; default: 0
	\item Initial protein content in particles.
	\item Input format: STARTPROT\quad$i$\quad$c_{i,1}\quad c_{i,2}\quad...\quad k_{i,NPROT}$
	\item $c_{i,j}$ is the amount of protein of type $j$ within particle type $i$
	\item Multiple lines should be used for different values of $i$.
\end{itemize}
%
\item {\it STARTDIR}
\begin{itemize}
	\item  type: float; size: \textit{NTYPE}; default: 0
	\item Initial transport state of particles. 1 = moving in the positive x-direction. -1 = moving in the negative x-direction. 0 = stationary.
	\item Input format: STARTDIR\quad$a_{1}\quad a_{2}\quad...\quad a_{NTYPE}$
	\item $a_{i}$ is the starting direction for particles of type $i$.
\end{itemize}
%
\item {\it USEPFUSE}
\begin{itemize}
	\item  type: logical; size: 1; default: F
	\item Determines whether fusion probability is used instead of a fusion rate.
	\item Input format: USEPFUSE value
\end{itemize}	
%
\item {\it VEL}
\begin{itemize}
	\item  type: float; size: \textit{NTYPE}; default: 0
	\item Velocity of different particle types.
	\item Input format: VEL\quad$v_{1}\quad v_{2}\quad...\quad v_{NTYPE}$
	\item $v_{i}$ is the velocity particles of type $i$.
\end{itemize}

% --------------------------

\end{itemize}

\bibliographystyle{aip} 
\bibliography{fiberModel}

\end{document}
